\usepackage{graphics}
\usepackage{tikz}
\usepackage{tikz-qtree}

\usetikzlibrary{arrows.meta, arrows, positioning}

\pgfmathsetmacro{\cardroundingradius}{15}

\pgfmathsetmacro{\cardwidth}{2.5}  % Card width in inches
\pgfmathsetmacro{\cardheight}{3.5} % Card height in inches

\pgfmathsetmacro{\cardexternalmargin}{0.1} % Margin to keep against the border

\pgfmathsetmacro{\cardwidthusable}{\cardwidth-2*\cardexternalmargin}  % Card width in inches
\pgfmathsetmacro{\cardheightusable}{\cardheight-2*\cardexternalmargin} % Card height in inches

\pgfmathsetmacro{\suspectrolewidth}{\cardwidth/2.5}  % Suspect role icon width
\pgfmathsetmacro{\suspectroleheight}{\suspectrolewidth} % Same formula as width, to keep 1:1 aspect ratio

\tikzset{
    mazearrow/.style={
        arrows={-Triangle[length=1.3mm,width=1.3mm]},
        dotted,
        line width=1pt,
        preaction={thin},
    },
}

\newcommand{\suspectrole}[1]{
    \node[anchor=north] (suspectrole) at (\cardwidth*.5 in, \cardheight in -\cardexternalmargin in) {
        \includegraphics[width=\suspectrolewidth in,height=\suspectroleheight in]{img/#1}
    };
}

\newcommand{\suspecttext}[1]{
    \node[anchor=south west,text width=\cardwidth*.25 in] at (\cardexternalmargin in, \cardheight in -\cardexternalmargin in-\suspectroleheight in) {
        \MakeUppercase{#1}
    };
    \node[xscale=-1,anchor=south west,text width=\cardwidth*.25 in] at (\cardwidth in - \cardexternalmargin in, \cardheight in -\cardexternalmargin in-\suspectroleheight in) {
        \MakeUppercase{#1}
    };
}

\newcommand{\cardtitle}[1]{
    \node[anchor=north] (cardtitle) at (suspectrole.south) {
        \Large \MakeUppercase{#1}
    };
}

\newcommand{\cardmaze}[1]{
    \node[below=of cardtitle.center]{
        \includegraphics[width=\cardwidthusable in,height=\cardheightusable in,keepaspectratio]{img/mazes/#1}
    };
}

\newcommand{\cardmazesol}[6]{
    \coordinate[below=0.5cm of cardtitle.center] (mazesol);

    \node[anchor=west] (a) at (\cardexternalmargin in, 0.5*\cardheight in) {#1};
    \node (b) at (\cardexternalmargin in + 0.2 * \cardwidthusable in, 0.5*\cardheight in) {#2};
    \node (c) at (\cardexternalmargin in + 0.4 * \cardwidthusable in, 0.5*\cardheight in) {#3};
    \node (d) at (\cardexternalmargin in + 0.6 * \cardwidthusable in, 0.5*\cardheight in) {#4};
    \node (e) at (\cardexternalmargin in + 0.8 * \cardwidthusable in, 0.5*\cardheight in) {#5};
    \node[anchor=east] (f) at (\cardexternalmargin in + \cardwidthusable in, 0.5*\cardheight in) {#6};

    \draw[mazearrow] (a.east) to (b.west);
    \draw[mazearrow] (b.east) to (c.west);
    \draw[mazearrow] (c.east) to (d.west);
    \draw[mazearrow] (d.east) to (e.west);
    \draw[mazearrow] (e.east) to (f.west);
    \draw[mazearrow, bend right=15] (f.north west) to (a.north east);
}

\newcommand{\cardmaintext}[1]{
    \node[anchor=north west,text width=\cardwidthusable in,align=left] at (\cardexternalmargin in, \cardheightusable*0.5 in) {
        \MakeUppercase{#1}
    };
}

\newenvironment{card}
{\begin{tikzpicture}\draw[rounded corners=\cardroundingradius] (0,0) rectangle (\cardwidth in, \cardheight in);}
{\end{tikzpicture}}