\usepackage{graphics}
\usepackage{tikz}
\usepackage{tikz-qtree}

\usetikzlibrary{arrows, positioning}

\pgfmathsetmacro{\cardroundingradius}{15}

\pgfmathsetmacro{\cardwidth}{2.5}  % Card width in inches
\pgfmathsetmacro{\cardheight}{3.5} % Card height in inches

\pgfmathsetmacro{\cardexternalmargin}{0.1} % Margin to keep against the border

\pgfmathsetmacro{\cardwidthusable}{\cardwidth-\cardexternalmargin}  % Card width in inches
\pgfmathsetmacro{\cardheightusable}{\cardheight-\cardexternalmargin} % Card height in inches

\pgfmathsetmacro{\suspectrolewidth}{\cardwidth/2.5}  % Suspect role icon width
\pgfmathsetmacro{\suspectroleheight}{\cardwidth/2.5} % Same formula as width, to keep 1:1 aspect ratio

\tikzset{
    %Define standard arrow tip
    >=stealth',
    %Define style for boxes
    punkt/.style={
           rectangle,
           rounded corners,
           draw=black, very thick,
           text width=6.5em,
           minimum height=2em,
           text centered},
    % Define arrow style
    pil/.style={
           ->,
           thick,
           shorten <=2pt,
           shorten >=2pt,}
}

\newcommand{\suspectrole}[1]{
    \node[anchor=north] (suspectrole) at (\cardwidth*.5 in, \cardheight in -\cardexternalmargin in) {
        \includegraphics[width=\suspectrolewidth in,height=\suspectroleheight in]{img/#1}
    };
}

\newcommand{\suspecttext}[1]{
    \node[anchor=south west,text width=\cardwidth*.25 in] at (\cardexternalmargin in, \cardheight in -\cardexternalmargin in-\suspectroleheight in) {
        \MakeUppercase{#1}
    };
    \node[xscale=-1,anchor=south west,text width=\cardwidth*.25 in] at (\cardwidth in - \cardexternalmargin in, \cardheight in -\cardexternalmargin in-\suspectroleheight in) {
        \MakeUppercase{#1}
    };
}

\newcommand{\cardtitle}[1]{
    \node[anchor=north] (cardtitle) at (suspectrole.south) {
        \Large \MakeUppercase{#1}
    };
}

\newcommand{\cardmaze}[1]{
    \node[below=of cardtitle.center]{
        \includegraphics[width=\cardwidthusable in,height=\cardheightusable in,keepaspectratio]{img/mazes/#1}
    };
}

\newcommand{\cardmazesol}[6]{
    \node[below=0.5cm of cardtitle.center] (mazesol) {};

    \node[anchor=south] (a) at (\cardwidthusable*0.16 in-\cardexternalmargin in, 0.5*\cardheight in) {#1};
    \node[right=0.2 in of a] (b) {#2}
    edge[pil,<-] (a.east);
    \node[right=0.2 in of b] (c) {#3}
    edge[pil,<-] (b.east);
    \node[right=0.2 in of c] (d) {#4}
    edge[pil,<-] (c.east);
    \node[right=0.2 in of d] (e) {#5}
    edge[pil,<-] (d.east);
    \node[right=0.2 in of e] (f) {#6}
    edge[pil,<-] (e.east)
    edge[pil,bend left=25] (a.south east);
}

\newcommand{\cardmaintext}[1]{
    % \node[below=1.2cm of mazesol.south east,text width=\cardwidthusable in-\cardexternalmargin in,align=left]  {
    %     \MakeUppercase{#1}
    % };
    \node[anchor=north west,text width=\cardwidthusable in-\cardexternalmargin in,align=left] at (\cardexternalmargin in, \cardheightusable*0.4 in) {
        \MakeUppercase{#1}
    };
}

\newenvironment{card}
{\begin{tikzpicture}\draw[rounded corners=\cardroundingradius] (0,0) rectangle (\cardwidth in, \cardheight in);}
{\end{tikzpicture}}