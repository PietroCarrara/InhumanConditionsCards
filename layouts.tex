\usepackage{graphics}
\usepackage{tikz}
\usepackage{tikz-qtree}

\usetikzlibrary{positioning}

\pgfmathsetmacro{\cardroundingradius}{15}

\pgfmathsetmacro{\cardwidth}{2.5}  % Card width in inches
\pgfmathsetmacro{\cardheight}{3.5} % Card height in inches

\pgfmathsetmacro{\cardexternalmargin}{0.1} % Margin to keep against the border

\pgfmathsetmacro{\cardwidthusable}{\cardwidth-\cardexternalmargin}  % Card width in inches
\pgfmathsetmacro{\cardheightusable}{\cardheight-\cardexternalmargin} % Card height in inches


\pgfmathsetmacro{\suspectrolewidth}{\cardwidth/2.5}  % Suspect role icon width
\pgfmathsetmacro{\suspectroleheight}{\cardwidth/2.5} % Same formula as width, to keep 1:1 aspect ratio

\newcommand{\suspectrole}[1]{
    \node[anchor=north] (suspectrole) at (\cardwidth*.5 in, \cardheight in -\cardexternalmargin in) {
        \includegraphics[width=\suspectrolewidth in,height=\suspectroleheight in]{img/#1}
    };
}

\newcommand{\suspecttext}[1]{
    \node[anchor=south west,text width=\cardwidth*.25 in] at (\cardexternalmargin in, \cardheight in -\cardexternalmargin in-\suspectroleheight in) {
        \MakeUppercase{#1}
    };
    \node[xscale=-1,anchor=south west,text width=\cardwidth*.25 in] at (\cardwidth in - \cardexternalmargin in, \cardheight in -\cardexternalmargin in-\suspectroleheight in) {
        \MakeUppercase{#1}
    };
}

\newcommand{\cardtitle}[1]{
    % \node[below=of suspectrole.south,scale=2] (cardtitle) {\MakeUppercase{#1}};
    \node[anchor=north,scale=2] (cardtitle) at (suspectrole.south) {\MakeUppercase{#1}};
}

\newcommand{\cardmaze}[1]{
    \node[below=of cardtitle.center]{
        \includegraphics[width=\cardwidthusable in,height=\cardheightusable in,keepaspectratio]{img/mazes/#1}
    };
}

\newenvironment{card}
{\begin{tikzpicture}\draw[rounded corners=\cardroundingradius] (0,0) rectangle (\cardwidth in, \cardheight in);}
{\end{tikzpicture}}